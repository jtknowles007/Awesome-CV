%!TEX root=../resume2.tex
%-------------------------------------------------------------------------------
%	SECTION TITLE
%-------------------------------------------------------------------------------
\cvsection{\faBriefcase\space\space Work Experience}


%-------------------------------------------------------------------------------
%	CONTENT
%-------------------------------------------------------------------------------
\begin{cventries}

%---------------------------------------------------------
  \cventry
    {Adjunct Instructor - Anatomy \& Physiology}
    {Harrison College}
    {Anderson, IN}
    {Apr. 2018 - PRESENT}
    {
      \begin{cvitems}
      \item {Develop course objectives, syllabus, lesson plans, tests, and lab activities for Anatomy \& Physiology I course} 
      \item {Correlate outcomes with career and employer expectations}
        \item {Instruct adult education students using a combination of in-class and online methods}
        \item {Engage students in active learning activities to encourage enthusiasm, discovery, and understanding of subject matter}
      \item {Utilize critical thinking exercises to encourage reasoning and investigative skills}
      \item {Assess student progress, identify at-risk learners, and aid development of problem solving and goal setting skills to ensure student success}
      \item {Provide individualized attention to discourage student withdrawal and maintain retention}
      \item {Handle sensitive and confidential student issues with tact and professionalism}
      \item {Maintain flexibility to meet the individual needs of students and campus}i
  \end{cvitems}
    }

%---------------------------------------------------------
  \cventry
    {Health Information Technology Specialist} % Job title
    {Qsource} % Organization
    {Indianapolis, IN} % Location
    {Dec. 2015 - PRESENT} % Date(s)
    {
      \begin{cvitems} % Description(s) of tasks/responsibilities
      \item {Facilitate data-driven beneficiary, community, and provider initiatives designed to increase patient safety, make communities healthier, and improve clinical quality}
      \item {Promote effective prevention and treatment of cardiovascular disease by partnering with physicians to provide more effective treatment}
        \item {Help providers prepare for and implement transition to value-based payments by participating in the Merit-based Incentive Payment System track of the Quality Payment Program}
    \item {Design individualized quality improvement initiatives using the Model for Improvement; monitor implementation, and provide ongoing assistance and support}
    \item {Train physician practices to use Electronic Health Records to their full potential and ensure that patients receive preventative cardiovascular health services}
    \item {Assist providers with successful improvement of care quality and use of healthcare IT to achieve system-wide efficiency, reduce readmissions, and lower costs}
    \item {Encourage provider reporting on measures that assess clinical quality of care, care coordination, patient safety, and patient and caregiver experience of care}
    \item {Collaborate with a diverse array of partners and stakeholders made up of local, state, and federal healthcare organizations}
    \item {Deliver cardiovascular health presentations to beneficiaries, caregivers, and providers at healthcare conferences and special events}
  \end{cvitems}
    }

%---------------------------------------------------------
  \cventry
    {Charge Nurse} % Job title
    {Fresenius Medical Care} % Organization
    {Indianapolis, IN} % Location
    {Feb. 2014 - Dec. 2015} % Date(s)
    {
      \begin{cvitems} % Description(s) of tasks/responsibilities
        \item {Provide specialized nursing care to patients with chronic renal failure utilizing home hemodialysis or peritoneal dialysis modalities}
        \item {Collaborate with interdisciplinary care team to assess, plan, and implement care plans to meet patient goals}
        \item {Monitor treatment regimen to ensure efficacy of treatment}
        \item {Assess patient and family learning readiness and provide appropriate education on renal disease, dialysis treatment, and medical regimen and its impact on the patient's health and wellbeing}
        \item {Instruct new patients, caregivers, and newly hired nurses on proper administration of peritoneal dialysis, aseptic technique, infection control, fluid status monitoring, and troubleshooting dialysis issues}
        \item {Administer medications and vaccines as ordered by physician}
        \item {Perform home visits upon initiation of home dialysis, and thereafter as needed}
        \item {Provide supervision and direction to nursing staff in accordance with policies, procedures, and guidelines}
        \item {Prepare monthly schedules for nursing staff and patients}
        \item {Review patient charts and electronic health record documentation for appropriateness of care}
        \item {Provide after hours and weekend on-call care for three separate clinics biweekly}
      \end{cvitems}
    }

%---------------------------------------------------------
  \cventry
    {Critical Care Staff Nurse} % Job title
    {Community Health Network} % Organization
    {Indianapolis, IN} % Location
    {Jul. 2012 - Feb. 2014} % Date(s)
    {
      \begin{cvitems} % Description(s) of tasks/responsibilities
        \item {Prioritize nursing care for assigned patients based on results of assessment and identified needs}
        \item {Assess critical care patient and analyze laboratory data to determine interventions needed}
        \item {Observe behavior and symptoms and report changes to on-call physicians}
        \item {Set up and monitor medical equipment and devices such as cardiac monitors, mechanical ventilators, oxygen delivery devices, transducers, and pressure lines}
        \item {Administer medications intravenously, by injection, orally, through gastric tubes, or other methods as appropriate}
        \item {Administer blood and blood products; monitor patient for signs and symptoms of transfusion reactions}
        \item {Draw or collect laboratory test as ordered}
        \item {Document patient's medical history and assessment findings in the electronic health record}
        \item {Collaborate with case managers, physicians, respiratory therapists, social workers, dieticians, physical/occupational therapists, pharmacists, patients, and family members to create and implement individualized plans of care}
        \item {Maintain ongoing communication with patient's family, caregivers, and the interdisciplinary team.}
        \item {Assist physicians with procedures such as endotracheal intubation, lumbar punctures, thoracentesis, insertion of chest tubes and central venous catheters}
        \item {Delegate tasks as appropriate to patient care technicians and student nurse externs}
        \item {Provide emergency care, CPR, and defibrillation as needed}
      \end{cvitems}
    }

  \cventry
    {EPIC Super User}
    {}
    {}
    {Jul. 2012 - Nov. 2013}
    {
      \begin{cvitems}
      \item {Assist EPIC trainer with classroom instruction prior to Go-Live}
      \item {Provide one-on-one support to physicians, nurses, and patient care technicians during implementation of EPIC electronic medical record system}
      \item {Troubleshoot problems as encountered and report critical issues to the EPIC Go-Live project team to ensure prompt resolution}
      \item {Promote the benefits of using the EPIC electronic medical record system and foster a positive experience during the transition period}
      \item {Continue long term support to users after completion of implementation}
      \item {Participate in regularly scheduled Super User, upgrade testing, and optimization meetings and communicate lessons learned back to users}
      \end{cvitems}
    }

  \cventry
    {Student Nurse Extern}
    {}
    {}
    {May 2011 - Jul. 2012}
    {
      \begin{cvitems}
      \item {Monitor and record vital signs, intake and output, and blood glucose measurements}
      \item {Provide ICU\/PCU patients with activities of daily living including ambulation, turns, toileting, bathing, feeding, and range of motion exercises}
      \item {Report changes in patient condition to staff nurses}
      \item {Assist with patient admission, transportation, transfer, and discharge}
      \item {Prepare recently vacated rooms for new patients}
      \item {Observe procedures and treatments performed by physicians, nurses, and respiratory therapists}
      \item {Perform CPR as needed}
      \end{cvitems}
    }

%---------------------------------------------------------
  \cventry
    {Life Skills Classroom Aide} % Job title
    {Hamilton Madison Shelby Educational Services} % Organization
    {Pendleton, IN} % Location
    {Aug. 2009 - Apr. 2012} % Date(s)
    {
      \begin{cvitems} % Description(s) of tasks/responsibilities
        \item {Oversee arrival and departure of students daily; assist students on and off school buses}
        \item {Assist students with feeding, toileting, and other life skills}
        \item {Instruct students individually and in small groups to achieve classroom objectives}
        \item {Adapt activities to support individual student goals}
        \item {Escort and assist students daily during interactions with general education population}
        \item {Assist students with personal care and mobility using apropriate assistive devices}
        \item {Provide input on development and revision of Individual Education Plan prior to annual meeting}
        \item {Supervise students in absence of teacher}
      \end{cvitems}
    }

%---------------------------------------------------------
\end{cventries}
