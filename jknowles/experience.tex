%!TEX root=../resume2.tex
%-------------------------------------------------------------------------------
%	SECTION TITLE
%-------------------------------------------------------------------------------
\cvsection{\faBriefcase\space\space Work Experience}


%-------------------------------------------------------------------------------
%	CONTENT
%-------------------------------------------------------------------------------
\begin{cventries}

%-------------------------------------------------------------------------------

  \cventry
    {Quality Improvement Consultant} % Job title
    {Indiana University Health} % Organization
    {Indianapolis, IN} % Location
    {Jun. 2020 - PRESENT} % Date(s)
    {
      \begin{cvitems} % Description(s) of tasks/responsibilities
        \item {Develop, initiate, and monitor outcomes of quality improvement initiatives for IU Health Plans Medicare Advantage HEDIS measures}
        \item {Increased Osteoporosis Management in Women STAR rating by 1.5 STARS through statewide collaboration with IU Health Pharmacists to provide post-fracture education and counseling}
        \item {Collaborate with multidisciplinary team to assist with implementation of IU Health Plans commercial line of business quality improvement data model and initial collection of commercial line of business supplemental data collection to determine baseline commercial quality scores}
        \item {Consult with IU Health and other external provider groups to assess and evaluate relevant data streams to develop and recommend strategies to improve quality of care for IU Health Plans Medicare Advantage members}
        \item {Review medical records from various data streams including IU Health internal data, IHIE, contracted provider medical records, and provider offices to abstract clinical data for HEDIS quality measure reporting}
        \item {Develop and update HCAHPS Power BI dashboards bi-annually to provide Quality Improvement, Population Health, and IU Health Plans senior management with data and visualizations of appropriate level of detail to facilitate strategic business and healthcare decisions and develop actionable and measurable quality improvement projects based on IU Health Plans member's perceptions of their healthcare experiences}
        \item {Train and advised healthcare quality and population health staff on HEDIS compliance}
  \end{cvitems}
    }


%---------------------------------------------------------
  \cventry
    {Quality Improvement Advisor / Health Information Technology Specialist} % Job title
    {Qsource} % Organization
    {Indianapolis, IN} % Location
    {Dec. 2015 - May 2020} % Date(s)
    {
      \begin{cvitems} % Description(s) of tasks/responsibilities
        \item {Initiate data-driven Medicare beneficiary and provider initiatives designed to increase patient safety and improve clinical quality across Indiana}
        \item {Advise providers during preparation and implementation phases of transition to value-based payment reporting through QPP MIPS}
        \item {Collect, query, and analyze data from recruited healthcare providers and Medicare claims}
        \item {Convert data into actionable insights by designing individualized quality improvement initiatives using the Model for Improvement}
        \item {Monitor implementation of quality improvement efforts, and provide ongoing support}
        \item {Assist providers with successful use of healthcare information technology to achieve system-wide efficiency, reduce readmissions, and lower costs}
        \item {Encourage provider reporting on measures that assess clinical quality of care, care coordination, and patient safety}
        \item {Facilitate community based multi-disciplinary care transition coalitions in North East and East Central Indiana}
        \item {Promote effective prevention and treatment of cardiovascular disease by providing educational resources and training to patients and providers}
        \item {Deliver healthcare presentations to beneficiaries, caregivers, and providers at healthcare conferences and special events}
  \end{cvitems}
    }

%---------------------------------------------------------
  \cventry
    {Adjunct Instructor}
    {Harrison College}
    {Anderson, IN}
    {Apr. 2018 - Sept. 2018}
    {
      \begin{cvitems}
      \item {Develop course objectives, syllabus, lesson plans, tests, and lab activities for Anatomy \& Physiology I, Anatomy \& Physiology II, and Health Information Management courses} 
        \item {Engage students in active learning activities to encourage enthusiasm, discovery, and understanding of subject matter}
        \item {Utilize critical thinking exercises to encourage reasoning and investigative skills}
        \item {Assess student progress, identify at-risk learners, and aid development of problem solving and goal setting skills to ensure student success}
        \item {Handle sensitive and confidential student issues with tact and professionalism}
        \item {Maintain flexibility to meet the individual needs of students and campus}
  \end{cvitems}
    }

%---------------------------------------------------------
  \cventry
    {Charge Nurse} % Job title
    {Fresenius Medical Care} % Organization
    {Indianapolis, IN} % Location
    {Feb. 2014 - Dec. 2015} % Date(s)
    {
      \begin{cvitems} % Description(s) of tasks/responsibilities
        \item {Provide specialized nursing care to patients with chronic renal failure utilizing home hemodialysis or peritoneal dialysis modalities}
        \item {Collaborate with interdisciplinary care team to assess, plan, and implement care plans to meet patient goals}
        \item {Monitor treatment regimen to ensure efficacy of treatment}
        \item {Assess patient and family learning readiness and provide appropriate education on renal disease, dialysis treatment, and medical regimen and its impact on the patient's health and well being}
        \item {Instruct new patients, caregivers, and newly hired nurses on proper administration of peritoneal dialysis, aseptic technique, infection control, fluid status monitoring, and troubleshooting dialysis issues}
        \item {Administer medications and vaccines as ordered by physician}
        \item {Perform home visits upon initiation of home dialysis, and thereafter as needed}
        \item {Provide supervision and direction to nursing staff in accordance with policies, procedures, and guidelines}
        \item {Prepare monthly schedules for nursing staff and patients}
        \item {Compile patient performance data and generate monthly reports to inform medical team and drive decisions to improve clinical outcomes}
        \item {Review patient charts and electronic health record documentation for appropriateness of care}
      \end{cvitems}
    }

%---------------------------------------------------------
  \cventry
    {Critical Care Staff Nurse} % Job title
    {Community Health Network} % Organization
    {Indianapolis, IN} % Location
    {Jul. 2012 - Feb. 2014} % Date(s)
    {
      \begin{cvitems} % Description(s) of tasks/responsibilities
        \item {Prioritize nursing care for assigned patients based on results of assessment and identified needs}
        \item {Assess critical care patients and analyze laboratory data to determine interventions needed}
        \item {Collaborated with multidisciplinary team to assist with implementation of IU Health Plans commercial line of business quality improvement data model and initial collection of commercial line of business supplemental data collection to determine baseline commercial quality scores}
        \item {Observe behavior and symptoms and report changes to on-call physicians}
        \item {Set up and monitor medical equipment and devices such as cardiac monitors, mechanical ventilators, oxygen delivery devices, transducers, and pressure lines}
        \item {Administer medications intravenously, by injection, orally, through gastric tubes, or other methods as appropriate}
        \item {Administer blood and blood products; monitor patient for signs and symptoms of transfusion reactions}
        \item {Draw or collect laboratory test as ordered}
        \item {Document patient medical history and assessment findings in the electronic health record}
        \item {Collaborate with case managers, physicians, respiratory therapists, social workers, dieticians, physical/occupational therapists, pharmacists, patients, and family members to create and implement individualized plans of care}
        \item {Maintain ongoing communication with patients, family members, caregivers, and the interdisciplinary team.}
        \item {Assist physicians with procedures such as endotracheal intubation, lumbar punctures, thoracentesis, insertion of chest tubes and central venous catheters}
        \item {Delegate tasks as appropriate to patient care technicians and student nurse externs}
        \item {Provide emergency care, CPR, and defibrillation as needed}
      \end{cvitems}
    }

  \cventry
    {EPIC Super User}
    {}
    {}
    {Jul. 2012 - Nov. 2013}
    {
      \begin{cvitems}
        \item {Assist EPIC trainer with classroom instruction prior to Go-Live}
        \item {Provide one-on-one support to physicians, nurses, and patient care technicians during implementation of EPIC electronic medical record system}
        \item {Troubleshoot problems as encountered and report critical issues to the EPIC Go-Live project team to ensure prompt resolution}
        \item {Promote the benefits of using the EPIC electronic medical record system and foster a positive experience during the transition period}
        \item {Continue long term support to users after completion of Go-Live}
        \item {Participate in regularly scheduled Super User, upgrade testing, and optimization meetings and communicate lessons learned back to users}
      \end{cvitems}
    }
%---------------------------------------------------------
\end{cventries}
\newpage
